\documentclass[./main]{subfiles}

\begin{document}


\section{Day 1}
In today's lecture, we will first review abelian duality in reprensetation theory, 
namely representation theory of finite abelian groups, as equivalences of categories of different levels. Then we will briefly review 
abelian duality in QFTs of different dimensions. 
Then we will see how to interpret the equivalences of categories as equivalences of field theories and state the main theorem.
Lastly, we will review the basics of topological field theories.


\subsection{Motivation: Representation theory}
Let us start right from the beginnign of representation theory:
$\mb{C}$-linear representations of finite abelian groups. 

Let $A$ be a finite abelian group, its Pontryagin dual group $\hat{A} \coloneqq Hom(A, \mbcx)$
is the group of characters of $A$. Its elements are group homomorphisms:
\begin{equation}
\chi : A \rightarrow \mbcx.
\end{equation}

If $X$ is a set, then we define $\mb{C}[X] \coloneqq Maps(X, \mb{C})$ to be the 
vector space of functions on $X$. Note that it is an algebra under point-wise multiplication:
\begin{equation}
    1_g \cdot 1_h \coloneqq \delta_{g,h} 1_g.
\end{equation}
% We will denote this multplication as $\cdot$.
Moreoever, if $G$ is a group, then $\mb{C[G]}$ has an additional multiplication structure:
\begin{equation}
    1_g  \ast 1_h \coloneqq 1_{gh}.
\end{equation}
$\mb{C[G]}$ together with $\ast$ product is called the group algebra of $G$.
Here's the first instance of abelian duality:
\begin{theorem}
There is a canoncial equivalence 
\begin{equation}
    \mb{C}[A] \isorightarrow \mb{C}[\hat{A}]
\end{equation}
that exchanges the point-wise multiplication $\cdot$ with group multiplication $\ast$.
\end{theorem}
On basis, this isomorphism takes 
\begin{equation}
    1_a \mapsto  \frac{1}{|A|}\sum_{\alpha} \alpha(a) 1_\alpha,
\end{equation}
where $\alpha$ sums over all elements of $\hat{A}$ (characters) and $\alpha(a) \in  \mbcx$ is the
evaluation of $\alpha$ at $a$. 

To get to our next example of abelian duality, let's take one step further and 
recall the classificiation theory for finite abelian groups:

Over $\mb{C}$, every finite dimensional A representation decomposes into direct sums 
irreducible ones. Since $A$ is abelian, the irreducible representations are all one-dimensional,
in fact, the equivalence classes of irreducible representations are precisely labeled by $\hat{A}$:
given a one dimension representation $V$ of $A$, after an identification $V \isorightarrow \mb{C}$,
we get a group homomorphism $A \rightarrow Aut(\mb{C}) = \mbcx$, aka a character.
For a general representation $V$, we see that they decomposes vector spaces over $\hat{A}$.


This can be expressed much more clearly as equivalence of categories:
\begin{theorem}
    Let $Rep_A$ be the category of finite dimensional $\mb{C}$-linear representations of $A$, 
    and $Vect_{\hat{A}}$ the category of finite dimensional vector bundles over $\hat{A}$ (a topological spaces with discrete topology).
    Then we have an equivalence of categories:
    \begin{equation}
        Rep_A \cong Vect_{\hat{A}}.
    \end{equation}
\end{theorem}

Since representations of $A$ are the same thing as vector bundles over $BA$, we can write the 
equivalence above as 
\begin{equation}
    Vect_{BA} \cong Vect_{\hat{A}}.
\end{equation}
\begin{remark}
Note that there is a symmetry between the $A$ side and the $\hat{A}$ side, unlike the 
previous equivalence. Also this is a statement of 1-categories and the iso of 
vector spaces is a statement of 0-categories
\end{remark}

Our last example is less well-known as the previous two but it is even more fundamental.
It is an equivalence of 2-categories. 

First we have to review group action on categories:
\begin{definition}
    Let $A$ be a finite abelian group and $C$ a category. An $A$ action on $C$ an assignment
    $F_a : C \rightarrow C$ for every $a \in A$, with associtivity data and higher coherences.
    Given a group $C$ with an $A$ action, then an equivariant object in $C$ is an object $c \in C$
    with the data $\phi_a: c \isorightarrow F_a(c)$ for every $a \in A$. 
    We denote $2Vect_{BA}$ the 2-category of 
    (small) $\mb{C}$-linear categories with a $A$ action.
\end{definition}

We have the following theorem:
\begin{theorem}[Freed-Teleman]
There is a canonical equivalence of 2-categories:
    \begin{equation}
        2Vect_{BA} \cong 2Vect_{B\hat{A}}.
    \end{equation}
It takes a category $C$ with an $A$ action to $C^A$ the category of $A$ equivariant object in $C$ (with a $\hat{A}$ action).
\end{theorem}

\begin{example}
Let $C = Vect$ with the trivial $A$ action, then $C^A \cong Rep_A$ and the $\hat{A}$ action is 
\begin{equation}
    F_a: V \mapsto V \otimes V_a.
\end{equation}
Let $C = Vect_A$ with the canonical $A$ action, then $C^A \cong Vect$ with the trivial $\hat{A}$ action.
In fact, these two example plus the theorem recovers $Vect_{BA} \cong Vect_{\hat{A}}$, even as categories with $\hat{A}$.
\end{example}

To summarize, we have three example of abelian duality: 
\begin{align}
    \mb{C}[A] &\cong \mb{C}[\hat{A}] \\
    Vect_{BA} &\cong Vect_{\hat{A}} \\ 
    2Vect_{BA} &\cong 2Vect_{B\hat{A}}.
\end{align}
There are two things to note:
\begin{itemize}
    \item There is an increase in category number. 
    \item To go up in category number, we have to add $B$ to one of the sides.
\end{itemize}

Let's try to understand this in a more field theoretic context. 

Topological abelian duality is a generalization of this for all category higher and more general object.
But before we get to that, we will first look at abelian duality in QFT.

\subsection{Abelian duailty in QFT}
Read chapter 1 of  \href{https://leon2k2k2k.github.io/Research/senior_thesis.pdf}{here} for not familiar with abelian duality in QFTs.
The main thing to get out is that there are equivalences of (free, higher)-abelian gauge theories in different dimensions:
\begin{center}
    \begin{tabular}{c c c}
        Dimension & theory & dual theory \\ 
        2 & Sigma model to $S^1$ & Sigma model to $S^1$ \\
        3 & $U(1)$ gauge theory & Sigma model to $S^1$ \\ 
        4 & $U(1)$ gauge theory & $U(1)$ gauge theory \\
    \end{tabular}
\end{center}
If we imagine $U(1)$ gauge theories are sigma models to $BU(1)$, then
notice it also exhibits the two behaviors that we observed above:
\begin{itemize}
    \item There is an increase in dimensions.
    \item To go one dimension higher, we need to put a $B$ on one side.
\end{itemize}
Moreoever, we know that this pattern continues on for arbitrary dimensions  and for general higher gauge theories:
\begin{theorem} \label{QFT theorem}
    In $d$ dimension, there is an equivalence of QFTs between the $n$-gerbe $U(1)$ theory
    and the $n-d-2$-gerbe $U(1)$ theory, with the coupling constant inverse proportional to each other.
\end{theorem}
The goal of this minicourse is to explore how we can take this QFT phenonemon to understand the abelian duality
we see in representation theory.

\subsection{Topological abelian duality}
Let's go back to the representation theory equivalences of categories. 
Through the study of (extended) topological field theories, we know that higher categories are directly related to 
topological field theories: $n$-categories are the values of $n+1$-dimensional topological field theories
at a point. In fact, equivalences of TFTs gives equivalences of $n$-categories. Thus if we can find topological versions
of \ref{QFT theorem}, then we will recover our rep. theory equivalences, in much greater generality.

This is precisely the main theorem:

\begin{theorem}
    Given a finite abelian group $A$ and an natural number $n$. In every dimension $d$,
    there is a $d$-dimensional topological field theory $Z_{B^nA}$ that counts $n$-gauge bundles on $A$.
    There is an equivalence of $d$-dimensional TFTs:
    \begin{equation}
        Z_{B^nA} \cong Z_{B^{d-n-1}\hat{A}} \otimes \mathcal{F},
    \end{equation}
    where $\mathcal{F}$ is an invertible field theory.
\end{theorem}

\begin{remark}
    This theorem can be extended to $\pi$-finite spectra, where the duality is called Brown-Comenetz duality.
    However, the proof we are using is designed to generalize to extended theories.
\end{remark}

\begin{remark}
    We will work in the unextended context, the search for the extended theories is still an open 
    research question. 
\end{remark}

The goal of this mini-course is to sketch a proof of the main theorem, but it's really not.
In order to prove this theorem in a conceptual way, we need to take a step back and 
understand the functoriality of field theories. Along the way,
we will encouter classical field theories, finite path integrals, correspondences, and sheaves theory.
In the end, we will see that this abstract, categorical approach will allow us to generalize
this to much much more general settings. That is the real main goal of this mini-course.

Tomorrow, we will start this journey by reviewing through the basics of topological field theories
and introduce the theories $Z_{B^nA}$ in the main theorem.



\end{document}
