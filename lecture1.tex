\documentclass[./main]{subfiles}

\begin{document}


\section{Day 1}
In today's lecture, we will first review abelian duality in reprensetation theory, 
namely representation theory of finite abelian groups, as equivalences of categories of different levels. Then we will briefly review 
abelian duality in QFTs of different dimensions. 
Then we will see how to interpret the equivalences of categories as equivalences of field theories and state the main theorem.
Lastly, we will review the basics of topological field theories.


\subsection{Motivation: Representation theory}
Let us start right from the beginnign of representation theory:
$\mb{C}$-linear representations of finite abelian groups. 

Let $A$ be a finite abelian group, its Pontryagin dual group $\hat{A} \coloneqq Hom(A, \mbcx)$
is the group of characters of $A$. Its elements are group homomorphisms:
\begin{equation}
\chi : A \rightarrow \mbcx.
\end{equation}

If $X$ is a set, then we define $\mb{C}[X] \coloneqq Maps(X, \mb{C})$ to be the 
vector space of functions on $X$. Note that it is an algebra under point-wise multiplication:
\begin{equation}
    1_g \cdot 1_h \coloneqq \delta_{g,h} 1_g.
\end{equation}
% We will denote this multplication as $\cdot$.
Moreoever, if $G$ is a group, then $\mb{C[G]}$ has an additional multiplication structure:
\begin{equation}
    1_g  \ast 1_h \coloneqq 1_{gh}.
\end{equation}
$\mb{C[G]}$ together with $\ast$ product is called the group algebra of $G$.
Here's the first instance of abelian duality:
\begin{theorem}
There is a canoncial equivalence 
\begin{equation}
    \mb{C}[A] \isorightarrow \mb{C}[\hat{A}]
\end{equation}
that exchanges the point-wise multiplication $\cdot$ with group multiplication $\ast$.
\end{theorem}
On basis, this isomorphism takes 
\begin{equation}
    1_a \mapsto  \frac{1}{|A|}\sum_{\alpha} \alpha(a) 1_\alpha,
\end{equation}
where $\alpha$ sums over all elements of $\hat{A}$ (characters) and $\alpha(a) \in  \mbcx$ is the
evaluation of $\alpha$ at $a$. 

To get to our next example of abelian duality, let's take one step further and 
recall the classificiation theory for finite abelian groups:

Over $\mb{C}$, every finite dimensional A representation decomposes into direct sums 
irreducible ones. Since $A$ is abelian, the irreducible representations are all one-dimensional,
in fact, the equivalence classes of irreducible representations are precisely labeled by $\hat{A}$:
given a one dimension representation $V$ of $A$, after an identification $V \isorightarrow \mb{C}$,
we get a group homomorphism $A \rightarrow Aut(\mb{C}) = \mbcx$, aka a character.
For a general representation $V$, we see that they decomposes vector spaces over $\hat{A}$.


This can be expressed much more clearly as equivalence of categories:
\begin{theorem}
    Let $Rep_A$ be the category of finite dimensional $\mb{C}$-linear representations of $A$, 
    and $Vect_{\hat{A}}$ the category of finite dimensional vector bundles over $\hat{A}$ (a topological spaces with discrete topology).
    Then we have an equivalence of categories:
    \begin{equation}
        Rep_A \cong Vect_{\hat{A}}.
    \end{equation}
\end{theorem}

Since representations of $A$ are the same thing as vector bundles over $BA$, we can write the 
equivalence above as 
\begin{equation}
    Vect_{BA} \cong Vect_{\hat{A}}.
\end{equation}
\begin{remark}
Note that there is a symmetry between the $A$ side and the $\hat{A}$ side, unlike the 
previous equivalence. Also this is a statement of 1-categories and the iso of 
vector spaces is a statement of 0-categories
\end{remark}

Our last example is less well-known as the previous two but it is even more fundamental.
It is an equivalence of 2-categories. 

First we have to review group action on categories:
\begin{definition}
    Let $A$ be a finite abelian group and $C$ a category. An $A$ action on $C$ an assignment
    $F_a : C \rightarrow C$ for every $a \in A$, with associtivity data and higher coherences.
    Given a group $C$ with an $A$ action, then an equivariant object in $C$ is an object $c \in C$
    with the data $\phi_a: c \isorightarrow F_a(c)$ for every $a \in A$. 
    We denote $2Vect_{BA}$ the 2-category of 
    (small) $\mb{C}$-linear categories with a $A$ action.
\end{definition}

We have the following theorem:
\begin{theorem}[Freed-Teleman]
There is a canonical equivalence of 2-categories:
    \begin{equation}
        2Vect_{BA} \cong 2Vect_{B\hat{A}}.
    \end{equation}
It takes a category $C$ with an $A$ action to $C^A$ the category of $A$ equivariant object in $C$ (with a $\hat{A}$ action).
\end{theorem}

\begin{example}
Let $C = Vect$ with the trivial $A$ action, then $C^A \cong Rep_A$ and the $\hat{A}$ action is 
\begin{equation}
    F_a: V \mapsto V \otimes V_a.
\end{equation}
Let $C = Vect_A$ with the canonical $A$ action, then $C^A \cong Vect$ with the trivial $\hat{A}$ action.
In fact, these two example plus the theorem recovers $Vect_{BA} \cong Vect_{\hat{A}}$, even as categories with $\hat{A}$.
\end{example}

To summarize, we have three example of abelian duality: 
\begin{align}
    \mb{C}[A] &\cong \mb{C}[\hat{A}] \\
    Vect_{BA} &\cong Vect_{\hat{A}} \\ 
    2Vect_{BA} &\cong 2Vect_{B\hat{A}}.
\end{align}
There are two things to note:
\begin{itemize}
    \item Each one is one category higher than the previous one. In fact, each one recovers the previous one as hochschild homology.
    \item Going from one to the next, we have to add $B$ to one of the sides.
\end{itemize}

Let's try to understand this in a more field theoretic context. 

Topological abelian duality is a generalization of this for all category higher and more general object.
But before we get to that, we will first look at abelian duality in QFT.

\subsection{Abelian duailty in QFT}
For now, see chapter 1 of \href{https://leon2k2k2k.github.io/Research/senior_thesis.pdf}{here}.

\end{document}
